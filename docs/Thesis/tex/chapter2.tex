\chapter{Requirement Analysis for a Patient Specific Planning-Software}

\section{Introduction}

A platform that is targeted to plan the thermal ablation procedure needs to fulfill a minimum set of features. The finished application should help the operator to plan the intervention. But besides the basic functional requirements there is also the non-functional features. As in all software projects, these are as important as functional features. Especially when the software is used in clinical practice, it has to meet usability and formal requirements. All organizational and formal requirements would go beyond the scope of this thesis and therefore will not be examined in detail. 
\\
In software engineering, a functional requirement is described as specific behavior of a software system, that defines what a system is supposed to accomplish \cite{wiki-functional}. In contrast non-functional requirements defines how a system is supposed to be, e.g. qualities of a system \cite{wiki-nonfunctional}.

Dr. Wolfgang Schramm has reviewed the requirements on such a system in his PhD thesis and therefore the planning software is mostly based on the requirement analysis did by him. In his PhD thesis he discussed the problems between accurate forecast of the treatment outcome and usability. Schramm describes the use of Finite Element Analysis (FEA), that simulates the treatment outcome of thermal ablation, but the additional medical benefit is still not studied \cite{diss-schramm}. Furthermore the detailed simulation requires high computing power, which leads to a very time-consuming process. That makes distinct forecast not feasible to perform during or immediately before the intervention \cite{diss-schramm}. 


\section{Functional Requirements}

\subsection{Image Analysis, Image Processing and Visualization}

Thermal ablation is an image-guided therapy and therefore imaging plays a main role in the application. Given that the simulation is based on the patient specific data, the application has to load the  data in order to simulate the intervention adequately. Since \gls{dicom} is an open standard, that is implemented by nearly all manufacturers of medical imaging devices, the application must load at least DICOM volumes. For further using, the DICOM volume has most likely to be post-processed. The physician needs to perform simple image processing algorithms to image registration and segmentation for high quality treatment planning \cite{diss-schramm}. 

The visualization of the structures is essential to view and interpret the outcome of the treatment. Due to this fact, visualization makes a major part of the functionality of the application. Not only the segmented structures, but also the 3D model of the applicator and the virtual resulting ablation zone have to be visualized. 

\subsection{Treatment Simulation}

The main function of the tool is to provide an adequate forecast of the treatment outcome. Since the resulting ablation zone is highly dependend on the used devices, it is necessary to use 3D models of the real devices for the simulation. Thus one of the main requirements is to make sure that the operator can use the 3D model devices with the same properties as the real ones. 

The actual ablation outcome is very hard to forecast due to the perfusion mediated temperature change during the ablation \cite{diss-schramm}. As mentioned in the introduction of this chapter, FEA can be used to simulate the outcome accurately, however this time-consuming technique is not feasible for clinical practice. Instead, the application must be able to visualize the ablation zone based on the estimated values for the coagulation zone dimensions from the respective vendor or from literature \cite{diss-schramm}. Depending on the ablation technique the shape of the ablation zone must also considered for more accurate estimation. Regarding the estimation of the resulting ablation zone, the simulation on basis of data from the vendor might not be as accurate as the FEA approach, but this is considered accurate enough to provide valuable information in the planning phase \cite{diss-schramm}. 

Ablation techniques like CA use multiple applicators which results in overlapping ablation zones. This must also be considered during the planning phase. The application has to be able to place multiple probes and ablation zones. Afterwards, the visibility of the single ablation zones need to be controllable for the operator.
Additionally the operator must be able to select the exact entry points on the skin and the targeted points within the tumor. 

After placing the probes and drawing the ablation zones, the operator must be provided with visual feedback. Because of preceding segmentation he now can see if for example a bone structure lies within the trajectory path or if large vessels lie within the ablation zone.

\section{Non-functional Requirements} 

The planning platform could either be used in clinical or research environment. In both scopes it must fulfill some formal requirements. By Austrian law, software is declared as a medical product and therefore it must fulfill all regulations and can only be used in a clinical environment when it is CE (Conformit� Europ�enne) certified. These certification processes are associated with additional quality assurance requirements and development costs. Since this application is a prototype, it will disregard these requirements.

\section{Conclusion}

For creating a planning platform that considers the patient's specific anatomy, it has to take the requirements described in this chapter into account. Considering these requirements the platform has to support several technologies. Instead of developing a standalone application, the software framework Slicer 3D was chosen as the basic application. Using Slicer avoids many unnecessary work, since it already implements the basic requirements and further it allows to integrate own functionality. 

Those requirements that have to be implemented as a module for Slicer are specified as user stories (see figure \ref{fig:userstories}).

\begin{figure}[htb]
\centering
\includegraphics[width = 120mm]{images/userstories.pdf}
\caption{Requirements on the platform specified as user stories.}
\label{fig:userstories}
\end{figure}