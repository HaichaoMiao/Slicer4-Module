\chapter*{Introduction}

Thermal Ablation as a treatment for localized tumors has been established continuously over the past years. Malignant liver tumors are one of the most common forms of cancer in the world, but unfortunately chemo and radiation therapy are ineffective against liver tumors \cite{dodd-2000}. Due to the fact, that many patients are no surgical candidates, thermal ablation techniques provide complementary treatment to eliminate or shrink tumors in a minimal invasive way and promise significant lower risk of major complications. There are numerous literature that describes the use of thermal ablation for the treatment of nearly all cancer types. 
\\
Medical imaging modalities have enabled the use of thermal energy to kill tumors without surgically cut the patient open. In the literature, these thermal ablation interventions are categorized as minimal invasive techniques. 
The interconnection between medical practice and technology has become tighter, as advances in technology were made. The intervention is therefore oftentimes not performed by one physician, but by a collaboration of several specialists. They are repeatedly referred as operators\footnote{see \cite{poon-rf}}. The complex procedure of thermal ablation and the necessity of collaboration suggest that  many benefits can be drawn from a planning-software, that enables outcome simulation on basis of the patient's specific anatomy.
\\ 
The development of a planning-software needs both knowledge of the thermal ablation techniques and the application of various software technologies. 
The architectural design of the planning-software will built upon software libraries and frameworks of third parties. These software technologies must be studied in detail. Accordingly, one major part of this thesis concentrates on the development process of the planning-software. 

