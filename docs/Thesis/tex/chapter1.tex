%patient selection, indicators etc. see dodd.pdf
% see dodd.pdf for additional infos

\chapter{Image-guided Thermal Ablation Techniques}
\section{Introduction}

\subsection{Overview}
Depending on the modalities, different ablation technologies are used in clinical environment. This chapter compares the different thermal ablative techniques in tumor therapy that are commonly used. 
\\
Although the rapid advancement in ablative technique has taken place within the past 20 years, experiments with (RF) ablation of tissues has been already documented at the end of the 19th century \cite{filingery2005}. 

What these ablative techniques have in common is, they attempt to kill every viable malignant cell within the designated area by doing minimal damage to the surrounding organs using thermal energy sources in a minimally invasive manner. There are several ways to accomplish this, but they can generally divided into two groups, based on whether heating or freezing is utilized to kill tumor cells \cite{ahmed-basic}. The most common one currently is the Radiofrequency Ablation technique, but other technologies are coming more and more to use. Cryoablation is used effectively in the treatment of prostate and kidney. 
Other thermal energy sources are Laser, High-intensity Focused Ultrasound and Microwave. Especially the latter has shown high potential in treating pulmonary tumors due to the significant larger ablation zone compared with RF \cite{brace-microwave}. 

\\
Chemical Ablation was documented being efficient in destroying hepatocellular carcinoma (HCC). Thereby chemical ablative substances (e.g. ethanol or acetic acid) are injected into the tumor  \cite{ahmed-basic}. These techniques will not be in the main focus of this thesis.

\subsection{Image Guidance and Minimal Invasiveness}

Tumor ablation represents the non-surgical alternative solution for patients with cancer. Especially with the rapidly increasing advancement in imaging technologies, the techniques became lesser invasive. Even during an intraoperative procedure imaging techniques, such as intraoperative ultrasound (IOUS), are necessary. Thus imaging is essential during the ablation process because the interventional radiologist needs to target the tumor inside the patients body without performing surgery (due to minimal invasiveness). 

% Image monitoring
% Postinterventional...

% (also in planning and post.. controlling)

% need better phrasing
\subsection{Advantages of Thermal Ablation}
Many advantages of ablative therapies exist when compared to surgical resection. 
\begin{itemize}
\item Minimal invasiveness. Small probes are punctuated through the skin into the center of the tumor. Due to imaging technologies, there is no need to cut through the skin in most cases. 
\item Compared to surgery, there are less complications reported during and after the procedure. % citation needed
\item Re-treatment possible. This is especially relevant when metastases reappear. 
\item Less tissue is removed compared to surgical resection, because only the tumor and a safety margin of 1cm has to be ablated.
\item Alternative to surgery. Patients who are not surgical candidates or Patients with few small Tumors are qualified for thermal ablation. 
\end{itemize}


% ... not alone by doctor. radiologist + assistance of technologies

% procedure

\section{Overview: Thermal Tumor Ablation}

Thermal ablative techniques differ mainly by their method of generating heat or cold. 
%need more here

\subsection{Radiofrequency Ablation}

% zitat bitte auf original �ndern
Since the early 1990 there has been enormous development of percutaneous Radiofrequency technology \cite{helmberger-rfa}. Radiofrequency Ablation (RFA) as a treatment for primary and secondary hepatic malignancies has been performed successfully for more than 10 years. But there has also been reports for RFA treatment for nearly all kinds of tumors. 

\subsubsection{Physical Background}

In RFA a high-frequency alternating current (typically between 450 - 500 kHz) is applied through an applicator to the tissue. Thereby an electric field within the tissue is established which oscillates with the applied radio frequency inducing ionic friction. When enough energy is deployed over a certain amount of time, the ionic friction results in loss of heat energy, which emerges into the tissue. A coagulative necrosis follows. 

% figure shows the interdependency of temperature and radius. 
% charring

The efficacy of the RFA and the resulting size of the coagulation zone depend on the following factors \cite{helmberger-rfa}: 

\begin{itemize}
\item amount of energy deployed
\item duration of exposure
\item probe design
\item tissue-specific factors (heat connectivity and conversion)
\item "heat sink" and "oven effect" %explain?
\end{itemize}

\subsubsection{Electrodes}

As mentioned before, the RF current is deployed an electrode directly into the tissue. In general these electrodes consists of an isolated shaft and an active tip. Figure shows various electrodes pursuing creating different coagulation volume shapes between 2 and 5 cm in diameter. 

%Figure

\subsubsection{Radiofrequency Generators}

The ablation process is controlled over the generator. Concerning the above mentioned tissue-specific and other factors, the amount and duration of energy exposure has to be well adapted for the tumor. The process is thereby very device-dependent. Thus the personal experience of the operator is essential to ensure efficient treatment of the disease \cite{poon-rf}.

% Figure

\subsubsection{Procedure}

RFA is usually performed under conscious sedation. Ultrasound (US), computed tomography (CT) and magnetic resonance imaging (MRI) are normally used for targeting the probe into the tumor and monitoring the result. It usually takes more than one placement, especially when several tumors has to be targeted. 

These imaging methods has both advantages and disadvantages. US is mostly used worldwide, because a radiologist is not required for the procedure. 
CT is used by interventional radiologist the most \cite{helmberger-rfa}. It has anatomically exact imaging and is widely available. MRI guidance has high contrast tumor-to-tissue but has the necessity for MR-compatible equipment \cite{diss-schramm}.

RFA is perceived by some clinicians as a simple treatment form, where simply just by inserting a needle and "cooking" the tumor \cite{poon-rf}. As showed above, the efficacy is dependent on many factors.  

%complications (maybe a summary for all techniques) 
%majo clinic

\subsubsection{Conculusion}

Studies has shown Radiofrequency Ablation to be an effective, safe and low-risk technique for treating liver tumors. It has gained more and more popularity over the past years. 

RFA is a technology-based treatment form and efficient application is thereby highly dependent on the experience of the operator. Due to the treatment's interdependence on various factors and it's highly conjunction with technology, it can be assumed that a planning phase is necessary. 

\subsection{Cryoablation}

The destruction of tissue by freezing is one of the oldest methods of tissue destruction known to mankind \cite{mortele-mri}. But substantial progress in destroying cancer tissue has been made not until recently. 
In the 60s, nitrogen-cooled probes for cryotherapy established in hepatic surgery. Probes were relatively large in size and open surgical was necessary for the placement of the probe \cite{silverman-cryo}.. The development of smaller percutaneous cryoprobes, using argon gas eliminated the risks associated with open surgery \cite{ahmed-basic}.
Currently Cryoablation (CA) is been successfully used to treat several types of tumor. 

\subsubsection{Physical Background}

In order to treat effectively and have control over the process the operator has to understand the mechanisms of cryogenic injury. In recent years freezing has not only used to destroy tissue, but also to preserve. The cell destruction with freezing is achieved by two major mechanisms. Cells are injured by ice crystal formation and the microcirculary failure occurs in the thawing period \cite{gage-cryo}. At low freezing rates, the freezing propagates through the extra-cellular space, which causes water to be drawn from the cell and cell results in osmotic dehydration \cite{ahmed-basic}. 
At faster freezing rates, intra-cellular ice crystal formation causes lethal damages to organelles and membrane \cite{ahmed-basic}.

\subsubsection{Devices}

As mentioned before, the development of argon-based systems replaced liquid nitrogen units, because argon has major advantages to nitrogen. As an example, Argon-based systems circulate very fast and as a result, iceball formation emerge very quickly. Probe tips of these systems reach temperatures around -150�C. 

The two cryoablation systems available in the United States are CRYOcare(TM), EndoCare inc., Irvine, CA and CryoHit(R)(TM), Seed Net(TM), Galil Medical, Wallingford, CT \cite{sharon-cryo}. The latter is the only unit with MRI compatible cryoprobes. They are both argon-based and use simultaneously multiple probes for the placement in the shape of the tumor in order to cover it. 

\subsubsection{Probes}

Both of above mentioned systems use eight sharp-tipped cryoprobes, that can directly place into the tumor. CRYOcare(TM) uses 2.4-4.9mm OD probes and CryoHit(R)(TM) 3.0mm probes. 

\subsection{Microwave Ablation}

Microwave Ablation (MWA), the latest development in tumor ablation, has been shown to have potential advantages over RFA, which is the most extensively applied modality. Zones of ablation are significantly larger and therefore faster treatment time can be achieved, compared to RFA. Yet, there is no approved device for patient treatment within the USA or Europe but in the Asian region there has been developing devices since the early 90s  \cite{boss-mwa}.

\subsubsection{Physical Background}

In Microwave Ablation tissue heating and resulting cell death is induced by electromagnetic waves in the high-frequency in the GHz order. Microwaves has wavelengths between infrared light and radio waves. 
Similar to RFA, an microwave antenna is placed into the tumor and emits electromagnetic microwaves into the surrounding tissue. It causes water molecules with an electric dipole moment to align themselves to the alternating electric field induced by microwaves. These oscillations of water molecules inside the cells results in warming. Macromolecules are not effected by microwaves but they are heated by convection nonetheless resulting in coagulation necrosis.

Although water molecules has a resonance frequency of ca. 22,2 GHz, they typically absorb 50\%-60\% of electromagnetic energy effectively in the range of 1-2 GHz \cite{boss-mwa}. Hence newly developed MWA devices work at frequencies below 1 GHz and use several applicators at the same time. 

\subsubsection{Devices}

As mentioned before, MWA devices for patient treatment has only applied in Asia so far. ref{fig:microtaze} (left) shows a japanese MW generator with generating power of 150 W. The chinese device (UMC-I Ultrasound-Guided Microwave Coagulator, Institue 207 Aerospace Industry Company, Beijing, China, and Department of Ultrasound of Chinese PLA General Hospital, Beijing, China) with generating power up to 80 W \cite{lu-mwa, boss-mwa}. Both operates at 2.450 MHz emission frequency. 


\begin{figure}[htb]
\centering
\includegraphics[width = 80mm]{images/microtaze.png}
\caption{Microtaze (Heiwa, Osaka, Japan) and needle electrodes, \cite{dodd-2000}}
\label{fig:microtaze}
\end{figure}

\begin{figure}[htb]
\centering
\includegraphics[width = 80mm]{images/umc.png}
\caption{"Multiple electrode insertion technique. Five guiding needles are inserted, then microwave energy is applied with one needle at a time." \cite{lu-mwa}}
\label{fig:umc}
\end{figure}

\subsubsubsection{Electrodes}

The japanese system uses electrodes with 1.6 mm in diameter and a 2-cm active tip, the chinese system 1.6 mm with a 2.7-cm tip. 

\subsection{Laser Ablation}

\subsection{Ultrasound}

\subsection{Conclusion}

Thermal Ablation Techniques have been developed within the last years as technological advances were made. There is a tight ...
% eine enge verbindung zwischen technologie und behandlungsmethoden.

\section{Decision Support for Performing Thermal Ablations}
\subsection{Image-guided Procedure}
\subsection{Workflow Description}
