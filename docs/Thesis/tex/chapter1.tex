%patient selection, indicators etc. see dodd.pdf
% see dodd.pdf for additional infos

\chapter{Image-guided Thermal Ablation Techniques}
\section{Introduction}

Different ablation methods are used in clinical environment. The decision which method is going to be used not only depends on the tumor location, but also on the geographical region where the treatment is applied. This chapter compares the different thermal ablative techniques in tumor therapy.
\\
Although the rapid advancement in ablative techniques has taken place within the past 20 years, experiments with tissue ablation using \gls{rf} has been already documented at the end of the 19th century \cite{filingery2005}. 

What all ablative techniques in general have in common is that, they attempt to kill every viable malignant cell within the designated area using thermal energy sources in a minimally invasive manner. At the same time minimal damage to the surrounding organs is done. There are several ways to accomplish this, but they can generally be divided into two groups, based on whether heating or freezing is utilized to kill tumor cells \cite{ahmed-basic}. Currently the most common method is Radiofrequency Ablation, but other techniques are coming more and more to use. Cryoablation is used effectively in the treatment of prostate and kidney cancer. On the other hand Microwave Ablation (MWA) has shown high potential in treating pulmonary tumors due to the significant larger ablation zone compared with RF \cite{brace-microwave}. Microwave Ablation is widely used in the Asian region.
\\
Other thermal energy sources are Laser and High-intensity Focused Ultrasound. 
Chemical substances can also be used to kill tumor cells. Chemical Ablation has been documented as being efficient in destroying \gls{hcc}. Thereby chemical ablative substances (e.g. ethanol or acetic acid) are injected into the tumor  \cite{ahmed-basic}. These techniques will although not be in the main focus of this thesis.

\subsection{Image Guidance and Minimal Invasiveness}

Tissue ablation with thermal energy sources (in literature also referred as tumor ablation) represents the non-surgical alternative solution for patients with cancer. Especially with the rapidly increasing advancement in imaging technologies, the techniques has become lesser invasive. Even during an intraoperative procedure imaging techniques, such as intraoperative ultrasound, are necessary. Hence imaging is essential during the ablation process, since the interventional radiologist utilizes it to target the tumor inside the patients body. Depending on the imaging modality and tumor properties, surgery is often not required. 

\begin{figure}[htb]
\centering
\includegraphics[width = 80mm]{images/liver-us.jpg}
\caption{Tumor ablation intervention under Ultrasound-guidance with CT scans of the patient before and after the ablation. Image taken from \cite{dodd-2000}}
\label{fig:liver-us}
\end{figure}

% Image monitoring
% Postinterventional...

% (also in planning and post.. controlling)

% need better phrasing
\subsection{Benefits of Thermal Ablation}
The increasing application of thermal ablation demonstrates the benefits of thermal ablation, which are
\begin{itemize}
\item Minimal invasiveness. Small probes are punctuated through the skin into the center of the tumor (percutanous). Due to imaging technologies, there is no need to cut through the skin in most cases. 
\item Compared to surgery, there are less complications reported during and after the procedure. % citation needed
\item Re-treatment is possible. This is especially relevant when metastases reappear. 
\item Less tissue is removed compared to surgical resection, because only the tumor and a safety margin of 1cm have to be ablated.
\item Alternative to surgery. Patients who are not surgical candidates or patients with few small Tumors are qualified for thermal ablation. 
\end{itemize}


% ... not alone by doctor. radiologist + assistance of technologies

% procedure

\section{Thermal Ablation Techniques}

Thermal ablative techniques differ mainly by their method of generating heat or cold. 
%need more here

\subsection{Radiofrequency Ablation}

Since the early 1990s there has been enormous development of percutaneous Radiofrequency technology \cite{helmberger-rfa}. \gls{rfa} as a treatment for primary and secondary hepatic malignancies has been performed successfully for more than 10 years. But there has also been reports for RFA treatment for nearly all kinds of tumors. 

\subsubsection{Technique of Radiofrequency Ablation}

In this technique, a high-frequency alternating current (typically between 450 - 500 kHz) is applied through an electrode into the biological tissue. Thereby an electric field within the tissue is established which oscillates with the applied radio frequency inducing ionic friction. When enough energy is deployed over a certain amount of time, the ionic friction results in loss of heat energy and the tissue starts to heat up. After that, the heat propagates deep into the tissue by the phenomena of convection. A coagulative necrosis follows. 

% figure shows the interdependency of temperature and radius. 
% charring

The efficiency of the RFA and the resulting size of the coagulation zone depend on the following factors \cite{helmberger-rfa}: 

\begin{itemize}
\item amount of energy deployed
\item duration of exposure
\item electrode design
\item tissue-specific factors (heat connectivity and conversion)
\item "heat sink" and "oven effect"
\end{itemize}

The so-called "heat sink" effect describes the influence of large vessels on lesion generation. If the targeted tumor is adjacent to large vessels, the heat is transferred away from the targeted tissue by the blood flow in the vessel \cite{heatsink-lu}. 
In literature the "oven effect" is described as effect of heat disposition caused by low thermal conductivity of background tissues. This effect is used to increase the heating efficacy \cite{oveneffect-livraghi}. 
The both above mentioned effects can also applied to other ablation methods. 

\subsubsection{Electrodes}

As mentioned before, the RF current is deployed through an electrode directly into the tissue. In general these electrodes consist of an isolated shaft and an active tip. Various electrodes create different coagulation volume shapes, mostly between 2 and 5 cm in diameter. 

\subsubsection{Radiofrequency Generators}

The ablation process is controlled over the generator. Concerning the above mentioned tissue-specific factors of the treatment efficiency, the amount and duration of energy exposure has to be well adapted for the tumor. The process is thereby very device-dependent. Thus the personal experience of the operator is essential to ensure efficient treatment of the disease \cite{poon-rf}.

% Figure

\subsubsection{Procedure}

RFA is usually performed under conscious sedation. \gls{us}, \gls{ct} and \gls{mri} are normally used for targeting the probe into the tumor and monitoring the result. It usually takes more than one placement, especially when several tumors have to be targeted. 

These imaging methods have both advantages and disadvantages. US is the most used worldwide, because with this setting no radiologist is required for the procedure. 
CT is the most favored imaging modality by interventional radiologists \cite{helmberger-rfa}. It has anatomically exact imaging and is widely available. Though MRI guidance has high contrast tumor-to-tissue, it has the necessity for MR-compatible equipment \cite{diss-schramm}. This circumstance does not exclude MRI, but therefore it is often the reason for using other imaging modalities. 

RFA is perceived by some clinicians as a simple treatment form, just by inserting a needle and "cooking" the tumor \cite{poon-rf}. As showed above, the procedure is more complex and the efficacy of the treatment is dependent on many factors. See figure \ref{fig:rfa-strategies} for different ablation strategies. 

\begin{figure}[htb]
\centering
\includegraphics[width = 130mm]{images/rfa-strategies.jpg}
\caption{The figures show the spherical thermal injuries done by the different RF ablation strategies. Tumors lesser sized than 2cm can be treated with a single ablation, larger tumors in size need to be performed multiple times \cite{dodd-2000}. Images taken from \cite{dodd-2000}}
\label{fig:rfa-strategies}
\end{figure}

%complications (maybe a summary for all techniques) 
%majo clinic

\subsubsection{Conclusion}

Studies has shown Radiofrequency Ablation to be an effective, safe and low-risk technique for treating liver tumors. It has come widely to use over the past years. 

RFA is a technology-based treatment form and efficient application is thereby highly dependent on the experience of the operator. Due to the treatment's interdependence on various factors and its highly conjunction with technology, support for the planning phase can be assumed as necessary

\subsection{Cryoablation}

The destruction of tissue by freezing is one of the oldest methods of tissue destruction known to mankind \cite{mortele-mri}. But substantial progress in destroying cancer tissue has been made not until recently. 
In the 60s, nitrogen-cooled probes for cryotherapy established in hepatic surgery. Probes were relatively large in size and open surgical was necessary for the placement of the probe \cite{silverman-cryo}. The development of smaller percutaneous cryoprobes, using argon gas, eliminated the risks associated with open surgery \cite{ahmed-basic}.
Currently \gls{ca} is successfully applied to treat several types of tumors. 

The major advantage of CA is the clear visualization of the iceball generation under different imaging modalities. This ensures precise monitoring and therefore better control of the ablation zone dimension. 

\subsubsection{Technique of Cryoablation}

In order to treat effectively and maintain control over the process the operator has to understand the mechanisms of cryogenic injury. In recent years freezing has not only been used to destroy tissue, but also to preserve. The cell destruction with freezing is achieved by two major mechanisms. Cells are injured by ice crystal formation and the microcirculary failure, which occurs in the thawing period \cite{gage-cryo}. At low freezing rates, the freezing propagates through the extra-cellular space, which causes water to be drawn from the cell and this results in osmotic dehydration \cite{ahmed-basic}. 
At faster freezing rates, the intra-cellular ice crystal formation causes lethal damages to organelles and membrane \cite{ahmed-basic}.

\subsubsection{Devices}

As mentioned before, the development of argon-based systems replaced liquid nitrogen units, because argon has major advantages to nitrogen. As an example, Argon-based systems circulate very fast and as a result, the iceball formation emerge very quickly. Probe tips of these systems reach temperatures around -150 �C. 

The two Cryoablation systems available in the United States are CRYOcare (EndoCare inc., Irvine, California, USA) and CryoHit (Galil Medical Ltd., Wallingford, CT, USA) \cite{sharon-cryo}. The latter is the only unit with MRI compatible cryoprobes \cite{cryohit}. They are both argon-based and use simultaneously multiple probes for the placement in the shape of the tumor in order to cover it. 

\subsubsection{Probes}

Both of above mentioned systems use eight sharp-tipped cryoprobes, that can directly placed into the tumor. CRYOcare uses 2.4-4.9mm OD probes and CryoHit 1.4-3.4mm probes. 

\begin{figure}[htb]
\centering
\includegraphics[width = 80mm]{images/cryo-tips.jpg}
\caption{The picture shows the ellipsoidal ice ball formation around the probe tip (Galil Medical, Yorkneam, Israel). Picture taken from \cite{silverman-cryo}.}
\label{fig:cryo-tips}
\end{figure}

\subsection{Microwave Ablation}

\gls{mwa}, the latest development in tumor ablation, has been shown to have potential advantages over RFA, which is the most extensively applied modality. Zones of ablation are significantly larger and therefore provide faster treatment, compared to RFA. Yet, there is no approved device for patient treatment within the USA or Europe but in the Asian region devices have been developed since the early 90s  \cite{boss-mwa}.

\subsubsection{Technique of Microwave Ablation}

In Microwave Ablation tissue heating and resulting cell death is induced by high-frequency electromagnetic waves in the GHz order. Microwaves have wavelengths between infrared light and radio waves. 
Similar to RFA, a microwave antenna is placed into the tumor and emits electromagnetic microwaves into the surrounding tissue. It causes water molecules with an electric dipole moment to align themselves to the alternating electric field induced by microwaves. These oscillations of water molecules inside the cells result in heating. Macromolecules are not effected by microwaves but they are heated by convection nonetheless resulting in coagulation necrosis.

Although water molecules have a resonance frequency of ca. 22,2 GHz, they typically absorb 50\%-60\% of electromagnetic energy effectively in the range of 1-2 GHz \cite{boss-mwa}. Hence newly developed MWA devices work at frequencies below 1 GHz and use several applicators at the same time. 

\subsubsection{Devices}

As mentioned before, MWA devices for patient treatment have only been applied in Asia so far. Figure \ref{fig:microtaze} (left) shows a japanese \gls{mw} generator with generating power of 150 W. The chinese device operates \footnote{UMC-I Ultrasound-Guided Microwave Coagulator, Institue 207 Aerospace Industry Company, Beijing, China, and Department of Ultrasound of Chinese PLA General Hospital, Beijing, China} with generating power up to 80 W \cite{lu-mwa, boss-mwa}. Both work at 2.450 MHz emission frequency. 

\begin{figure}[htb]
\centering
\includegraphics[width = 80mm]{images/microtaze.png}
\caption{Microtaze (Heiwa, Osaka, Japan) and needle electrodes. Taken from  \cite{dodd-2000}}
\label{fig:microtaze}
\end{figure}

\begin{figure}[htb]
\centering
\includegraphics[width = 80mm]{images/umc.png}
\caption{"Multiple electrode insertion technique. Five guiding needles are inserted, then microwave energy is applied with one needle at a time." Taken from \cite{lu-mwa}}
\label{fig:umc}
\end{figure}

\subsubsection{Electrodes}

The japanese system uses electrodes with 1.6 mm in diameter and a 2-cm active tip, the chinese system 1.6 mm with a 2.7-cm tip. Figure \ref{fig:umc} shows a procedure performed with an UMC-I device. 

\subsection{Laser Ablation}

\gls{litt} is another minimal invasive thermal ablative technique. High energy laser radiation is applied through optical fiber directly into the tumor and energy absorption leads to heating and consequently to tissue destruction. 
Since laser light is used instead of RF, MR imaging is compatible with LITT devices. This circumstance assures real-time monitoring of the status of the intervention, due to the good soft-tissue contrast and high spatial resolution of MR imaging. Another advantage is the preservation of a well-defined area of necrosis around the fiber tip and thus minimal damage is done to the surrounding tissue \cite{laser-mack}. 

Although LITT is a suitable technique for local tumor destruction within solid organs, it is mostly used for the destruction of primary and secondary liver tumors. 

\subsubsection{Technique of Laser Ablation}

Laser light with a wavelength between 1060 and 1200nm is transmitted through a MR-compatible probe to the tissue which leads to coagulation necrosis \cite{laser-vogl}. Photons of this wavelength have a deep penetration depth. The penetration depth of photons is not only depended on the wavelength, but also dependent on the tissue. The interaction between biological tissue and laser light is controlled by many physical phenomena, but at these small amount of energy, which is used in LITT, only scattering and absorption has effective influence on the process. Thanks to the effect of thermal conduction, the temperature extends into the tissue. 
The laser light itself is transmitted via optical fiber. The applicator holds magnetite markers to allow easier visualization and positioning during the intervention \cite{laser-mack}.

\subsubsection{Devices}

Most LITT devices use a neodynium: yttrium-aluminum-garnet (Nd:YAG) laser source. Devices differ in design, types and size of optic fibers, probe tip design and the number of applicators. 
The MediLas 5100 (Dornier MedTech, Germany) uses Nd:YAG lasers with a wavelength of 1064nm. The laser is delivered through a specially developed flexible diffusing applicator (figure \ref{fig:medilas5100} and \ref{fig:flexibleapplicator}).

\begin{figure}[htb]
\centering
\includegraphics[width = 80mm]{images/medilas5100.jpg}
\caption{Dornier MediLas 5100 laser system. Picture taken from \cite{img-litt2}}
\label{fig:medilas5100}
\end{figure}

\begin{figure}[htb]
\centering
\includegraphics[width = 50mm]{images/flexibleapplicator.jpg}
\caption{The MediLas 5100 flexible applicator. Picture taken from \cite{img-litt1}}
\label{fig:flexibleapplicator}
\end{figure}

\subsection{High-intensity Focused Ultrasound}

\gls{hifu} is currently studied as a potential method for noninvasive destruction of localized tumors. The main advantage of using HIFU is the noninvasiveness of this treatment. With a transducer ultrasound creates a focused energy beam from the distance. 
Ultrasound, as form of vibrational energy propagates through the tissue as a mechanical wave. At the targeted point the beams interfere in a hot spot and result in coagulation necrosis. Figure \ref{fig:hifu} shows the principle of the HIFU technique.

\begin{figure}[htb]
\centering
\includegraphics[width = 70mm]{images/HIFU.png}
\caption{Schematic of HIFU treatment. Ultrasound is bundled in a focal point, where it results in heating. Image taken from \cite{img-hifu}}
\label{fig:hifu}
\end{figure}


\section{Shapes of Ablation Zones}

Currently available probes provide ablation zone sizes between 2 and 5 cm in diameter. As discussed before, the effective resulting ablation depends on many factors. The following table shows a good approximation for different probes and electrode designs of various ablation devices. 

\begin{center}
\begin{tabular}{ | l | l | l | l |}
    \hline
	Probe & Number of Applicators & Shape & Technique \\ \hline
	Cool Tip & single needle\footnotemark[1] & ellipsoid\footnotemark[1] & RFA  \\ 
	Cool Tip Cluster & cluster electrodes\footnotemark[1] & ellipsoid\footnotemark[1] & RFA  \\ 
	LeVeen & multi-tined electrodes\footnotemark[1] & spherical\footnotemark[1] & RFA  \\ 
	Starburst XL & multi-tined electrodes\footnotemark[1] & pear\footnotemark[1] & RFA  \\ 
	Rita & perfused Talon needle\footnotemark[1] & pear\footnotemark[1] & RFA  \\ 
	Celon (Olympus) & single needle\footnotemark[1] & ellipsoid\footnotemark[1] & RFA  \\ 
	Microblate & single electrode & spherical\footnotemark[2]  & MWA   \\ 
	Microtaze & single electrode & ellipsoid\footnotemark[3]  & MWA   \\  % brochure
	IceSeed & single needles & spherical\footnotemark[4]  & CA   \\  % brochure
	IceSphere & single needles & ellipsoid\footnotemark[4] & CA   \\   % brochure
	IceRod & single needles & ellipsoid\footnotemark[4] & CA   \\   % brochure
	IceBulb & single needles & bulb\footnotemark[4] & CA   \\  \hline % brochure
\label{tab:shapes}
\end{tabular}

\end{center}
\footnotetext[1]{\cite{helmberger-rfa}}
\footnotetext[2]{\cite{mw-jones} }
\footnotetext[3]{\cite{microtaze}}
\footnotetext[4]{\cite{cryohit}}

\section{Requirement for a Planning Platform for Thermal Ablation}

Thermal Ablation Techniques are gaining popularity as a complementary choice of therapy for different tumors. Although it is commonly considered a safe and low-risk technique, it still suffers from a noticeable rate of complications. A review showed that the complication rate of RFA for liver tumors is higher than previously assumed \cite{mulier-rfa}. RFA treatments still have higher tumor recurrence rate than surgical resection. The procedure is more complex as it appears at first. It comprehends not simply the insertion of a needle and the "cooking" of the tumor \cite{poon-rf}.
The treatment outcome of an intervention is highly influenced by different factors:
\begin{itemize}
\item The placement of the applicator(s) is by far the most time-consuming part of the process. Tumors larger in size or multiple small tumors require repeated applicator placement and ablation.
However the accurate placement is essential for the efficient treatment. 
\item The imaging modalities for the guidance of the applicator placement and the monitoring of the developing ablation zone influences the treatment success. Ultrasound is the most frequently used modality, but it is inadequate for imaging the ablation zone accurately \cite{diss-schramm} 
\item Larger tumors require lager ablation zones, but this is accompanied by loss of control with an increased rate of damage to healthy tissue. 
\item Tumors adjacent to large vessels are more difficult to ablate completely since the blood flow acts like a heat sink (see heat sink effect).
\item Since thermal ablation techniques are technology-based treatments, the outcome is highly dependent on the wide variation of probe devices, imaging modalities and experience of the operator with these technologies. 
\end{itemize}

The accurate estimation of the ablation zone and treatment result is essential for the success of the intervention. Due to the fact that the above mentioned factors have great influence on the outcome of a thermal ablation intervention, it can be assumed that using a Planning Platform for Thermal Ablation Techniques will positively affect the outcome. This system estimates the ablation zone based on the patient-specific anatomy using 3D visualization and virtually place applicators. \\

\section{Conclusion}

Thermal ablation techniques have been developed within the last years, as technological advances were made. They are used in clinical practice for the treatment of different localized tumors. Shorter hospital stay, lesser complications during the procedure compared to surgical resection are only some of the advantages of these techniques.

Radiofrequency Ablation is the most commonly used technique, followed by Cryoablation and Microwave Ablation. Each technique has advantages and disadvantages. However there is still very little textbook knowledge for education and practice available \cite{diss-schramm}. In addition, the theoretical work of the simulation of the ablation zone is advanced (e.g. see \gls{fea}), but generic models are used instead of taking the patient's anatomic structure into account \cite{diss-schramm}. Due to these circumstances and the strong interconnection between medical practice and technology, it is most likely that benefits from treatment simulation can be derived. 
Therefore the next chapter is describing the requirements on such a planning platform. 

% eine enge verbindung zwischen technologie und behandlungsmethoden.
% generell geringe krankenhaus aufenthalt und weniger komplikationen
% malignant cells are more sensitive
% outcome prediction